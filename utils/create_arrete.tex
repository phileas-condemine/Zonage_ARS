\documentclass[]{article}
\usepackage{lmodern}
\usepackage{amssymb,amsmath}
\usepackage{ifxetex,ifluatex}
\usepackage{fixltx2e} % provides \textsubscript
\ifnum 0\ifxetex 1\fi\ifluatex 1\fi=0 % if pdftex
  \usepackage[T1]{fontenc}
  \usepackage[utf8]{inputenc}
\else % if luatex or xelatex
  \ifxetex
    \usepackage{mathspec}
  \else
    \usepackage{fontspec}
  \fi
  \defaultfontfeatures{Ligatures=TeX,Scale=MatchLowercase}
\fi
% use upquote if available, for straight quotes in verbatim environments
\IfFileExists{upquote.sty}{\usepackage{upquote}}{}
% use microtype if available
\IfFileExists{microtype.sty}{%
\usepackage{microtype}
\UseMicrotypeSet[protrusion]{basicmath} % disable protrusion for tt fonts
}{}
\usepackage[margin=1in]{geometry}
\usepackage{hyperref}
\hypersetup{unicode=true,
            pdftitle={Arrêté},
            pdfauthor={ARS},
            pdfborder={0 0 0},
            breaklinks=true}
\urlstyle{same}  % don't use monospace font for urls
\usepackage{graphicx,grffile}
\makeatletter
\def\maxwidth{\ifdim\Gin@nat@width>\linewidth\linewidth\else\Gin@nat@width\fi}
\def\maxheight{\ifdim\Gin@nat@height>\textheight\textheight\else\Gin@nat@height\fi}
\makeatother
% Scale images if necessary, so that they will not overflow the page
% margins by default, and it is still possible to overwrite the defaults
% using explicit options in \includegraphics[width, height, ...]{}
\setkeys{Gin}{width=\maxwidth,height=\maxheight,keepaspectratio}
\IfFileExists{parskip.sty}{%
\usepackage{parskip}
}{% else
\setlength{\parindent}{0pt}
\setlength{\parskip}{6pt plus 2pt minus 1pt}
}
\setlength{\emergencystretch}{3em}  % prevent overfull lines
\providecommand{\tightlist}{%
  \setlength{\itemsep}{0pt}\setlength{\parskip}{0pt}}
\setcounter{secnumdepth}{0}
% Redefines (sub)paragraphs to behave more like sections
\ifx\paragraph\undefined\else
\let\oldparagraph\paragraph
\renewcommand{\paragraph}[1]{\oldparagraph{#1}\mbox{}}
\fi
\ifx\subparagraph\undefined\else
\let\oldsubparagraph\subparagraph
\renewcommand{\subparagraph}[1]{\oldsubparagraph{#1}\mbox{}}
\fi

%%% Use protect on footnotes to avoid problems with footnotes in titles
\let\rmarkdownfootnote\footnote%
\def\footnote{\protect\rmarkdownfootnote}

%%% Change title format to be more compact
\usepackage{titling}

% Create subtitle command for use in maketitle
\newcommand{\subtitle}[1]{
  \posttitle{
    \begin{center}\large#1\end{center}
    }
}

\setlength{\droptitle}{-2em}

  \title{Arrêté}
    \pretitle{\vspace{\droptitle}\centering\huge}
  \posttitle{\par}
    \author{ARS}
    \preauthor{\centering\large\emph}
  \postauthor{\par}
      \predate{\centering\large\emph}
  \postdate{\par}
    \date{10/07/2019}


\begin{document}
\maketitle

ARRETE 1

Portant determination des zones caracterisees par une offre
insuffiSsante ou par des difficultes dans l'acces aux Soins concernant
la profesSion de medecin Le Directeur General de \textbar{}'Agence
Régionale de Sante Bretagne Vu le code de la Santé publique et notamment
Son article L. 1434-4 ; Vu la loi n* 2016-41 du 26 janvier 2016 de
modernisation de notre Système de Sante, notamment Son Article 158, Vu
le décret n* 2010-336 du 31 mars 2010 portant création des Agences
Régionales de Sante ; Vu le décret du 19 février 2015 portant nomination
de Monsieur Olivier de CADEVILLE en qualité de Directeur General de
l'Agence Régionale de Sante Bretagne ä compter du 9 mars 2015, Vu le
décret n* 2017-632 du 25 avril 2017 relatif aux conditions de
détermination des zones Caractérisées par une offre de Soins
insuffisante ou par des difficultés dans l'accès aux Soins ou dans
lesquelles le niveau de l'offre est particulièrement élevé ; Vu l'arrête
du 13 novembre 2017 relatif à la méthodologie applicable ä la profession
de médecin pour la détermination des zones prévues au 1° de l'article L.
1434-4 du code de la santé publique.

Vu l'avis de la conférence régionale de santé et de l'autonomie réunion
en sa séance plénière le XXXX pris conformément aux dispositions de
l'article R. 1434-42 du code de la Sante publique ;

Considérant les résultats de la concertation organisée au niveau
régional ; ARRÊTE

Article 1er : Le présent arrête abroge celui en date du XXXX, portant
adoption de la révision du projet régional de Sante Bretagne, dans 5a
partie relative ä la détermination des zones prévues à l'article L.
1434-4 du code de la Sante publique. Article 2: Les zones caractérisés
par une offre de Soins insuffisante ou par des difficultés dans l'accès
aux Soins concernant la profession de médecin sont arrêtées ainsi qu'il
suit en région Bretagne. Ces zones Sont reparties en trois catégories :

\begin{itemize}
\item
  les zones d'intervention prioritaire ;
\item
  les zones d'action complémentaire.
\item
  Les zones de vigilance.
\end{itemize}

La liste des communes, leur rattachement à un territoire de vie-Sante et
leur qualification est jointe en annexe 1 de cet arrête.

La cartographie de ce zonage figure en annexe 2 du même arrête.

Les territoires de vie-Sante sont qualifiés, dans un premier temps, par
la méthode nationale décrite en annexe de l'arrêté méthodologique du 13
novembre 2017, et, dans un second temps, par une méthodologie régionale
concertée qui tient compte des spécificités de la région et des données
d'offre de soins les plus actualisées (disponible via le lien Suivant :
XXXX).

La méthodologie régionale objective par un Score la Situation des
territoires de vie-Sante en fonction de trois dimensions l'accès à
l'offre de Soins actuelle, l'évolution à venir de l'offre et les
caractéristiques de la population, en termes d'état de Santé et de
niveau de précarité, de chaque territoire de vie-Santé. Article 3:
Conformément ä la possibilité qui II est offerte par les dispositions de
l'arrête méthodologique du 13 novembre 2017, le Directeur General de
l'Agence Régionale de Sante Bretagne a la possibilité de retenir les
territoires de vie-Sante dont l'indicateur APL ANNEE\_CALCUL\_APL est
supérieur ou égal à 4 consultations par an et par habitant, dans la
limite autorisée de 5 \% de la population du vivier régional, Soit XXXX
\% de la population de la région.

Dans ce cadre, NB\_ZIP territoires ont été requalifies en zones
d'intervention prioritaire et XXXX en zones d'action complémentaire.

Les territoires de vie-Santé ayant un APL ANNEE\_CALCUL\_APL supérieur à
4, sélectionnés en zones d'intervention prioritaire, sont LIST\_ZIP.

Ces NB\_ZIP territoires de vie-Sante ont une population résidente faible
(moins de 1000 habitants) et l'offre de Soins Sur place n'existe pas ou
repose sur un seul médecin. Cette grande fragilité de l'offre de Soins
sur des territoires isolés justifie leur sélection en zones
d'intervention prioritaire.

Les territoires de Vie-Santé ayant un APL ANNEE supérieur à 4,
sélectionnés en zones d'action complémentaire, sont de deux types :

\begin{itemize}
\tightlist
\item
  des territoires insulaires LIST\_ZAC\_INSULAIRES. Ces
  NB\_ZAC\_INSULAIRES territoires de vie-Sante ont une population
  résidente plus conséquente mais l'offre de soins repose sur peu de
  médecins et reste fragile au regard de la spécificité insulaire de ces
  deux territoires. Cette fragilité de l'offre de Soins Sur des
  territoires isolés du continent a justifié leur sélection en zones
  d'action complémentaire.
\item
  Les territoires de vie-Sante avec une offre de Soins qui s'est
  dégradée : LIST\_ZAC\_AUTRES La méthodologie régionale a ciblé
  plusieurs territoires de vie-Sante ayant un APL 2015 de plus de 4 avec
  un Score régional de fragilité élevé, reflétant une offre de Soins
  insuffisante. Dans la limite autorisée, NB\_ZAC\_AUTRES territoires de
  vie-Sante, les plus défavorisés au regard des critères retenus dans la
  méthode régionale, ont et qualifies en zones d'action complémentaire.
\end{itemize}

Les critères objectivant la situation de l'offre de Soins sur ces
territoires sont présentes en annexe 2 du présent arrêté.

Article 4 : Le présent arrêté peut faire l'objet, dans un délai de deux
mois à compter de la notification ou de la publication, soit :

\begin{itemize}
\item
  d'un recours grâcieux auprès du Directeur général de l'Agence
  Régionale de Sante ;
\item
  d'un recours hiérarchique auprès du Ministère des Solidarités et de la
  Santé ;
\item
  d'un recours contentieux devant le Tribunal administratif de Rennes ;
  Le recours gracieux ne conserve pas le délai des autres recours.
\end{itemize}

Article 5 : Le Directeur General de l'Agence Régionale de Sante Bretagne
est en charge de l'exécution du présent arrêté qui sera publié au
recueil des actes administratifs de la préfecture de la région Bretagne.


\end{document}
